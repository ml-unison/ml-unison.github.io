\documentclass[11pt]{article}

%% LaTeX Preamble - Common packages

\usepackage[utf8]{inputenc}
\usepackage[spanish]{babel}
\usepackage{graphicx}
\usepackage{amsmath,amssymb}
\usepackage[left=2cm,right=2cm,top=2cm,bottom=2cm]{geometry}

\spanishdecimal{.}

\title{Examen rápido No. 7}
\author{Reconocimiento de Patrones (2023-2)}
\date{Julio Waissman Vilanova} % delete this line to display the current date

%%% BEGIN DOCUMENT
\begin{document}

\maketitle

%\vspace{5mm}

\textbf{Nombre}: \line(1,0){400}

%\vspace{9mm}


\begin{enumerate}

\item Sea un nodo $\mathbf{A}$ en un árbol de decisión el cual tiene 35 ejemplos de la clase 1 y 43 ejemplos de la clase 2. El nodo tiene dos nodos hijos. El nodo hijo de la izquierda (nodo $\mathbf{B}$) consta de 5 ejemplos de la clase 1 y 25 ejemplos de la clase 2, mientras que el nodo hijo de la derecha (nodo $\mathbf{C}$) tiene 30 ejemplos de la clase 1 y 18 ejemplos de la clase 2. Calcular la entropía de cada nodo y la ganancia de información que se obtiene.

    \begin{center}
    \begin{tabular}{|c||p{5cm}|}
    \hline
    Entropía nodo $\mathbf{A}$ & \\
    \hline
    Entropía nodo $\mathbf{B}$ & \\
    \hline
    Entropía nodo $\mathbf{C}$ & \\
    \hline
    Ganancia de información & \\
    \hline
    \end{tabular}
    \end{center}
    
    \item Encierre en un circulo los que sean métodos de prepoda de árboles de decisión:

    \begin{center}
    \begin{tabular}{|p{8cm}|p{8cm}|}
    \hline
    Pocos ejemplos en el nodo padre  & Todos los ejemplos son de la misma clase en el nodo padre \\
    \hline
    La ganancia de información es negativa & La ganancia de información es un valor pequeño positivo \\
    \hline
    La mayoría de los ejemplos en el nodo padre pertenecen a la misma clase & Existe igual número de ejemplos de cada clase en el nodo padre\\
    \hline
    Los nodos hijos clasifican mal el conjunto de prueba & Un nodo hijo tiene mucho más ejemplos que los otros nodos hijos\\
    \hline
    La ganancia de información es nula & Un nodo hijo no tiene ejemplos\\  
    \hline
    \end{tabular}
    \end{center}
    
    \item (10 puntos) En 4 lineas de texto como máximo, indique cual es la idea del \emph{Bagging}

    \begin{center}
    \begin{tabular}{l p{16cm}}
    1 & \\
    \hline
    2 & \\
    \hline
    3 & \\
    \hline
    4 & \\
    \hline
    \end{tabular}
    \end{center}
    
    \item (10 puntos) Indique cual es la principal diferencia entre \emph{Bagging} y \emph{Boosting} en una sola línea (dos si tienes la letra grande).
    
    \begin{center}
    \begin{tabular}{l p{16cm}}
    1 & \\
    \hline
    2 & \\
    \hline
    \end{tabular}
    \end{center}
\end{enumerate}
\end{document}